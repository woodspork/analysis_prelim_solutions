\documentclass[11pt,twoside]{amsart}
\usepackage{amssymb, amsmath, enumerate, mathpazo, hyperref, mathtools}

\DeclarePairedDelimiter{\ceil}{\lceil}{\rceil}
\usepackage[normalem]{ulem}
\usepackage{fullpage}
\usepackage[T1]{fontenc}
\renewcommand{\labelitemi}{\guillemotright}
\usepackage{mathrsfs}
\usepackage{float}

\theoremstyle{plain}
\newtheorem{prob}{Problem}

\newcommand{\RR}{\mathbb{R}}
\newcommand{\ZZ}{\mathbb{Z}}
\newcommand{\CC}{\mathbb{C}}
\newcommand{\NN}{\mathbb{N}}
\newcommand{\QQ}{\mathbb{Q}}


\title{January 2024 Preliminary Exam}
\author{}

\begin{document}
\maketitle

\begin{prob}
  Suppose $\{a_n\}$ and $\{b_n\}$ are two complex sequences such that 
  \begin{equation*}
    \lim_{n \to \infty}{a_n b_n} = 0.
  \end{equation*}
  Show that at least one of $\{a_n\}$ and $\{b_n\}$ has a subsequences that converges to zero.
\end{prob}
\begin{proof}
  Consider the sequence here
\end{proof}

\begin{prob}
  Suppose $f: [a,b] \to \RR$ is bounded, and moreover $f$ is Riemann integrable on $[a,c]$
  for all $a < c < b$. Show that $f$ is Riemann integrable on $[a,b]$. 
\end{prob}
    
\newpage
\begin{prob}
  Suppose $f: \RR \to \RR$ is differentiable and $\lim_{x \to \infty} f'(x) = 0$. Show that 
  if the sequence $\{f(n)\}_{n \in \NN}$ converges, then the limit $\lim_{x \to \infty} f(x)$ exists. 
\end{prob}
\begin{proof}
  Let $L := \lim_{n \in \NN, n \to \infty} f(n)$. We will show $\lim_{x \to \infty} f(x) = L$.\\
  Let $\epsilon > 0$. Then there exists an $N \in \NN$ such that $|L - f(m)| < \epsilon/2$ for all 
  $m \geq N$. Similarly, there exists an $M \in \NN$ such that $|f'(p)| < \epsilon/2$ for all 
  $p \geq M$. Set $P = \max\{ N, M \}$. Then for all $x \geq P$ set $\tilde{x} = \ceil{x}$
  so that $\tilde{x} \in \NN$ and $\tilde{x} \geq P$ as well. Then 

  \begin{align*}
    |L - f(x)| &= |L - f(x) - f(\tilde{x}) + f(\tilde{x})|\\
               &\leq |L - f(\tilde{x})| + |f(\tilde{x}) - f(x)| \text{ by the triangle inequality,}\\
               &< \epsilon/2 + |f(\tilde{x}) - f(x)| \text{ since } \tilde{x} \geq P. 
  \end{align*}

  It remains to show that $|f(\tilde{x}) - f(x)| < \epsilon/2$. Since $f$ is differentiable, by the 
  Mean Value Theorem, there exists $c \in (x, \tilde{x})$ such that 
  
  \begin{equation*}
    \frac{|f(\tilde{x}) - f(x)|}{|\tilde{x} - x|} = |f'(c)| < \epsilon/2 
  \end{equation*}
  where the inequality holds since $c \geq P$. Then, 

  \begin{equation*}
    |f(\tilde{x}) - f(x)| <  |\tilde{x} - x|\epsilon/2.
  \end{equation*}

  Putting it altogether, we get the required bound 

  \begin{equation*}
    |L - f(x)| < \epsilon/2 + |f(\tilde{x}) - f(x)| < \epsilon/2 + \epsilon/2 = \epsilon. 
  \end{equation*}

  This shows that the limit $\lim_{x \to \infty} f(x)$ equals $L$, and hence, exists. 
\end{proof}

\newpage
\begin{prob}
  A function $f: \RR \to \RR$ is called Lipschitz if there exists $M > 0$ such that 
  \begin{equation*}
    |f(x) - f(y)| \leq M|x-y|
  \end{equation*}
  for all $x,y \in \RR$. Show that every Lipschitz function on $\RR$ is uniformly continuous on $\RR$,
  but not every uniformly continuous function on $\RR$ is necessarily Lipschitz on $\RR$. 
\end{prob}
\end{document}


















