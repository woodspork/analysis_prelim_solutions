\documentclass[12pt]{article}
\usepackage{amssymb, amsmath, enumerate, hyperref}

\usepackage[normalem]{ulem}
%\usepackage{fullpage}
%\usepackage[T1]{fontenc}
%\renewcommand{\labelitemi}{\guillemotright}
%\usepackage{mathrsfs}
\usepackage{float}
\usepackage{capstone}

\theoremstyle{plain}
\newtheorem{prob}{Problem}

\title{January 2017 Preliminary Exam}
\author{Committee Members: No record}
\date{Solutions: Abigail Ciasullo, Usman Hafeez, Adam O'Neal, \& Emma Pickard}

\begin{document}
\maketitle

\begin{center}
{\bf\Large Advanced Calculus, Mandatory Problems}
\end{center}

\begin{enumerate}
    \item Suppose that $f:\R \rightarrow \R$ is differentiable and $f'$ is continuous.
    Show that the restriction of $f$ to any closed interval $[a,b]$ is Lipschitz continuous.

    \item Suppose that $(X,d)$ is a metric space, fix a point $a$, and let $f(x) = d(a,x)$.
    Show that the function $f: X \rightarrow \R$ is uniformly continuous.
\end{enumerate}

\begin{center}
{\bf\Large Advanced Calculus, Optional Problems}
\end{center}

\begin{enumerate}
    \setcounter{enumi}{2}
    \item Let $f:(a,b) \to \R$ be differentiable.
    Suppose that $f'(c) = 0$ for some $c\in (a,b)$.
    Show that if $f$ has a local minimum at $x=c$ and $f''(c)$ exists, then $f''(c) \geq 0$.

    \item Suppose that $f\in C[0,1]$ and 
    \[
        \int_a^b{x^nf(x)dx} = 0 \textrm{\hspace{12pt} for all integer } n\geq 0.
    \]
    Show that $f$ is the zero function.
\end{enumerate}
\newpage

\begin{center}
{\bf\Large Real Analysis, Mandatory Problems}
\end{center}
\setcounter{enumi}{0}
\begin{enumerate}
    \item For $f \in L^1(\R)$, the Fourier transform of $f$ is defined by
    \[
        \hat{f}(\xi) = \int_{\R}{f(x)e^{-2\pi ix\xi}dx}.
    \]

    \begin{enumerate}
    \item Show that $\hat{f}$ is continuous in $\R$.
    \item Show that $\hat{f} \rightarrow 0$ as $\card{\xi} \rightarrow \infty$.
    For this part, you may assume that the set of all linear combinations of characteristic functions over bounded open intervals is dense in $L^1(\R)$.
    \end{enumerate}
    
    \item Let $f$ be an integrable function in $\R^d$.
    Show that
    \[
        \lim_{\alpha\rightarrow\infty}{\alpha m\left\{x\in\R^d : \card{f(x)}>\alpha\right\}} = 0.
    \]
\end{enumerate}

\begin{center}
    {\bf\Large Real Analysis, Optional Problems}
\end{center}
\begin{enumerate}
    \setcounter{enumi}{2}
    \item \begin{enumerate}
    \item State Egorov's Theorem.
    \item Use Egorov's Theorem to prove the Bounded Convergence Theorem: if ${f_k}$ is a sequence of measurable functions on a measurable set $E$ with $m(E) < \infty$, such that $f_k \rightarrow f$ a.e. in $E$ and $\card{f_k} \leq M$ a.e. in $E$ for some finite constant $M$, then
    \[
        \int_E{\card{f_k-f}dx}\rightarrow \textrm{\hspace{12pt} as } k\rightarrow\infty.
    \]
    \end{enumerate}
    
    \item For a measurable function $f$ on a measurable set $E\subset\R$, define
    \[
        \| f \|_{L^{\infty}(E)} = \inf\left\{\alpha : m\left\{x\in E : \card{f(x)}>\alpha\right\} = 0\right\}.
    \]
    Show that if $\| f \|_{L^{\infty}(E)} < \infty$ and $0 < m(E) < \infty$, then
    \[
        \lim_{p\rightarrow\infty}\left(\frac{1}{m(E)}\int_{E}{\card{f}^pdx}\right)^{\frac{1}{p}} = \| f \|_{L^{\infty}(E)}.
    \]
\end{enumerate}

\newpage
\vspace*{\fill}
\begin{center}
    {\bf\Huge Solutions}
\end{center}
\vspace*{\fill}

\newpage
\begin{center}
    {\bf\Large Advanced Calculus, Mandatory Problems}
\end{center}

\begin{enumerate}
    \item Suppose that $f:\R \rightarrow \R$ is differentiable and $f'$ is continuous.
    Show that the restriction of $f$ to any closed interval $[a,b]$ is Lipschitz continuous.

    \begin{proof}
        For a function to be Lipschitz continuous, we must have that $\card{f(x)-f(y)} \leq M\card{x-y}$ for some constant $M$.
        Since $f$ is continous and differentiable, in particular on $[a,b]$, the Mean Value Theorem implies that for any $x,y\in [a,b]$, there exists some $c\in [a,b]$ such that
        \[
            f'(c) = \frac{f(x)-f(y)}{x-y}.
        \]
        Applying the absolute value to each side, we obtain that $\card{f(x)-f(y)} = \card{f'(c)}\card{x-y}$.

        Now, since $f'$ is continuous, in particular on $[a,b]$, we know that $f'$ must attain some maximum, call it $M$, on $[a,b]$.
        Thus we have
        \[
            \card{f(x)-f(y)} = \card{f'(c)}\card{x-y} \leq M\card{x-y}.
        \]
        Therefore, $f$ is Lipschitz continuous on any closed interval $[a,b]$.
    \end{proof}

    \newpage\item Suppose that $(X,d)$ is a metric space, fix a point $a$, and let $f(x) = d(a,x)$.
    Show that the function $f: X \rightarrow \R$ is uniformly continuous.

    \begin{proof}
        A function $f$ is uniformly continuous if for all $\eps > 0$, there exists $\delta >0$ such that for all $x,y\in X$, $\card{f(x)-f(y)} < \eps$ whenever $\card{x-y} < \delta$.
        Since $X$ is a metric space, we know by the triangle inequality and symmetry that $d(a,x) \leq d(a,y) + d(x,y)$.
        This implies that $d(a,x)-d(a,y) \leq d(x,y)$.
        
        Choose $x,y\in X$ so that $d(x,y) < \delta$ and set $\eps = \delta$. 
        Then we have
        \[
            \card{f(x) - f(y)} = \card{d(a,x)-d(a,y)} \leq d(x,y) < \delta = \eps.
        \]
        Thus, $f$ is uniformly continuous.
    \end{proof}

\newpage
\begin{center}
    {\bf\Large Advanced Calculus, Optional Problems}
\end{center}    

    \item Let $f:(a,b) \to \R$ be differentiable.
    Suppose that $f'(c) = 0$ for some $c\in (a,b)$.
    Show that if $f$ has a local minimum at $x=c$ and $f''(c)$ exists, then $f''(c) \geq 0$.

    \begin{proof}
        Since $f$ has a local minimum at $x = c$,  we know that $c$ is a critical point of $f$.
        Furthermore, since we know that $f''$ exists, we may apply the second derivative test which states that for any critical point $x_0$, if $f''(x_0) > 0$, then $f$ has a local minimum at $x_0$ and if $f''(x_0) < 0$ then $f$ has a local maximum at $x_0$.
        The test is inconclusive if $f''(x_0) = 0$.

        Now, since $c$ is a critical point of $f$, if $f''(c) < 0$, the second derivative test would imply that $f$ has a local maximum at $x = c$, which we know is false.
        Note that if $f''(c) = 0$, the test is inconclusive and so doesn't directly contradict that $f$ has a local minimum at $c$.
        Thus, it must be that $f(c) \geq 0$.
    \end{proof}

    \newpage\item Suppose that $f\in C[0,1]$ and 
    \[
        \int_a^b{x^nf(x)dx} = 0 \textrm{\hspace{12pt} for all integer } n\geq 0.
    \]
    Show that $f$ is the zero function.

    \begin{proof}
        
    \end{proof}
\end{enumerate}

\newpage
\newpage
\begin{center}
    {\bf\Large Real Analysis, Mandatory Problems}
\end{center}
\begin{enumerate}
    \item For $f \in L^1(\R)$, the Fourier transform of $f$ is defined by
    \[
        \hat{f}(\xi) = \int_{\R}{f(x)e^{-2\pi ix\xi}dx}.
    \]

    \begin{enumerate}
    \item Show that $\hat{f}$ is continuous in $\R$.
    
        \begin{proof}
            Since $f\in L^1(\R)$, we know that $f$ is integrable.
            That is, $\int_{\R}{f(x)} < \infty$.
            Furthermore, since $f$ is integrable, $\card{f}$ is as well.
            
            Consider some sequence $\xi_n \rightarrow \xi$ and let $g_n(x) = f(x)e^{-2\pi ix\xi_n}$.
            Observe that this is a measurable function for all $n$ as it is the product of two measurable functions.
            Furthermore, $g_n(x) \rightarrow f(x)e^{-2\pi ix\xi} \eqqcolon g(x)$ for all $x\in\R$.
            Next, note that $\card{f(x)e^{-2pi ix\xi_n}} = \card{f(x)}$.
            Thus, we have that $g_n(x) \leq \card{f(x)}$ where $f$ is an integrable function and $g_n$ is a sequence of measurable functions with $g_n(x)\rightarrow g(x)$.
            So by the Dominated Convergence Theorem, we know that
            \[
                \int_{\R}{f(x)e^{-2\pi ix\xi_n}dx} \rightarrow \int_{\R}{f(x)e^{-2\pi ix\xi}dx}.
            \]
            That is, $\hat{f}(\xi_n) \rightarrow \hat{f}(\xi)$.
            Recall that a function $h$ is continuous if and only if it takes convergent sequences to convergent sequences.
            In other words, if $x_n \rightarrow x$, then $h(x_n) \rightarrow h(x)$.

            Therefore, since $\xi_n \rightarrow \xi$ and we have that $\hat{f}(\xi_n) \rightarrow \hat{f}(\xi)$, it follows that $\hat{f}$ is continuous.
        \end{proof}

    \item Show that $\hat{f} \rightarrow 0$ as $\card{\xi} \rightarrow \infty$.
    For this part, you may assume that the set of all linear combinations of characteristic functions over bounded open intervals is dense in $L^1(\R)$.
    \end{enumerate}
\end{enumerate}

\end{document}


















